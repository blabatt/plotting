\section{GNUPlot}
\textit{General script pattern involves an arbitrary number of }\textbf{set}\textit{ commands for setup of axes, titles, fonts, and other styles; followed by one or more }\textbf{plot}\textit{ commands, each of which renders output to the default or designated \say{terminal}. See \href{http://gnuplot.info/demos/}{here} for numerous demos, or in <src>/demo directory.}


%%%%%%%%%%%%%%%%%%%%%%%%%%%%%%%%%%%%%%%%%%
\subsection*{Invocation}
\code{gnuplot [options]* [-c script-file]?}\\[2mm]
\textit{GNUPlot is invoked in either batch mode (script given to STDIN); or in interactive mode, which exposes a prompt. Default settings can be overridden at startup by initialization files in either <install>/.gnuplotrc or \textasciitilde /.gnuplot. These files can invoke normal }\textbf{set}\textit{ commands as well as setting environment variables such as: } \texttt{GNUTERM}, \texttt{GDFONTPATH}, \texttt{GNUPLOT\_DEFAULT\_GDFONT}. \\[2mm]
\textit{Interactive (command-line) commands include:} 
{\footnotesize 
\begin{tabular}{l l l l}
    break       & cd            & call      &   clear   \\
    contin.     & do            & evaluate  &   exit    \\      
    \href{http://www.gnuplot.info/demo/fit.html}{\textbf{fit}}  & help        & history   & if        \\
    for         & import        & load      & lower     \\
    pause       & \textbf{[s]plot}  & print & printerr  \\
    pwd         & quit          & raise     & refresh   \\
    replot      & reread        & reset     & save      \\
    \textbf{set}  & show        & shell     & stats     \\
    system      & test          & toggle    & undef.    \\
    unset       & update        & vclear    & vfill     \\
    while
\end{tabular}}


\begin{comment}
\textit{Command line options include:}\
\begin{itemize}[label=-]
    \item V \quad \ul{version}
    \item h \quad \ul{help}
    \item p \quad \ul{persist} plot windows after close 
    \item d \quad \ul{default-settings}
    \item s \quad \ul{slow}: wait for font initialization
    \item e \quad execute given command
    \item c \quad call given following script
\end{itemize}
\end{comment}



%%%%%%%%%%%%%%%%%%%%%%%%%%%%%%%%%%%%%%%%%%
\subsection*{Plotting} 
\code{[s]plot [<\ul{p}lot\ul{e}lement>]+}\\[2mm]
%\code{[<data> using <map> with <style>]*}
\textbf{replot} \textit{ and }\textbf{refresh}\textit{ repeat the last }\textbf{plot}\textit{ command. PE has the following elaborate structure:}

{\footnotesize 
\begin{itemize}[label=|-]
\item \textbf{axes} \dots to display $\in$ []
\item <data> \dots $\in$ <file>|\textquotesingle \textquotesingle|\textquotesingle -\textquotesingle|\textquotesingle +\textquotesingle|<\say{sampled}>
    \begin{itemize}[label=|-]
	\item \textbf{csv} \dots datafile is in csv format
	\item \textbf{index} \dots [m, [n, [p]]] select from datasets
	\item \textbf{every} \dots periodic row selection
	\item \textbf{skip} \dots used to skip col headers
	\item \textbf{using} \dots map data to plotting cols
	    \begin{itemize}[label=|-]
		\item \textbf{key} \dots row-based titles
    	\item \textbf{bins} \dots segregate x into buckets
    	\item \textbf{smooth} \dots interpolation $\in$: \\
    	   unique | frequency | cumulative | bins | kdensity | [a|m|]csplines | [s]bezier | unwrap | zsort
		\item \textbf{{[x|x2|y|y2|cb]}ticlabels} \dots 
	    \end{itemize}
    \end{itemize}
\item \textbf{title} \dots [\textbf{columnheader}[(N)]] <title> [\textbf{at} \dots]
\item \textbf{with} \dots see Plot Types section
\end{itemize}}
\vspace{2mm}

\textit{Plot examples:}\\
\entry{35mm}{plot sin(x)}{\say{sampled} <data>}\\
\entry{35mm}{plot \textquotesingle -\textquotesingle <{}< EOD \dots}{inline \say{here-doc}}\\
\entry{40mm}{plot for [j=1:3] sin(j*x)}{for clause}\\
\entry{40mm}{plot for [i in var] i.\textquotedbl .dat\textquotedbl}{for-in syntax}\\
\entry{40mm}{plot \textquotedbl file\textquotedbl \, with lines,}{with clause}\\
\entry{40mm}{\phantom{xxx}\textquotedbl file2\textquotedbl \, with points}{second PE}\\
\entry{40mm}{plot [-pi:pi*2] [-5:5] tan(x)}{x, y ranges}\\
\entry{40mm}{plot [t=1:10] tan(t), x**t}{\href{http://www.gnuplot.info/demo/param.html}{parametric}}\\
\code{plot \textquotedbl file\textquotedbl using (tan(\$2)):(\$3) smooth csplines \textbackslash}\\
\code{\phantom{xxx}axes x1y2 title \textquotedbl parametric\textquotedbl with lines}\\
 
\textbf{using}\textit{ phrase maps }<data>\textit{ to the abscissa \& ordinate (or angle and radius). For example:}\\
\entry{40mm}{\dots using 1:(\$2+\$3)}{derived fields}\\
\entry{40mm}{\dots using 1:5:6 \textquotesingle \%lf,\%lf,\%lf\textquotesingle}{comma sep}\\
\entry{40mm}{\dots using 1:(\$3>7 ? \$2 : 1/7)}{ternary filtered}\\
\entry{40mm}{\dots using 2:4:xtic(1):ytic(3)}{labels}\\



%%%%%%%%%%%%%%%%%%%%%%%%%%%%%%%%%%%%%%%%%%
\subsection*{Set Command}


%%%%%%%%%%%%%%%%%%%%%%%%%%%%%%%%%%%%%%%%%%
\subsection*{Script Syntax}

\textit{Expressions:}\\
\entry{35mm}{var1 + var2}{variables} \\
\entry{35mm}{sin(x) + 2}{function call} \\
\entry{35mm}{\{1.0, -3.4\}}{constant $\in\mathbb{C}$} \\
\entry{35mm}{\textasciitilde var}{one\textquotesingle s comp.} \\
\entry{35mm}{var1 \&\& var2}{logical AND} \\[2mm]

\textit{Operators:}
{\footnotesize\begin{tabular}{l l l l l}
    **      & *         & /     & \%        &
\end{tabular}}
\vspace{2mm}

\textit{Iteration:}\\

\textit{Stringss:}\\

\textit{Command-line substitution:}\\

\entry{35mm}{\# \dots}{comment}


%%%%%%%%%%%%%%%%%%%%%%%%%%%%%%%%%%%%%%%%%%
\subsection*{Built-in Functions}

\textit{Math Functions (data-type default is $\mathbb{C}$ then $\mathbb{R}$):}\\
{\footnotesize
\begin{tabular}{l l l l}
    abs         & acos[h]       & airy      & arg       \\
    asin[h]     & atan[2|h|]    & Elliptic  & besj[0|1|n]       \\
    besy[0|1|n] & besi[0|1|n]   & ceil      & cos[h]    \\
    erf[c]      & exp           & expint    & floor     \\
    gamma       & ibeta         & inverf    & igamma    \\
    imag        & invnorm       & int       & lambertw  \\
    lgamma      & log[10]       & norm      & rand      \\
    real        & sgn           & sin[h]    & sqrt      \\
    tan[h]      & vogt
\end{tabular}}\vspace{2mm}

\textit{String, Data, \& Date Functions:}\\
{\footnotesize
\begin{tabular}{l l l l}
    column          & col.head      & exists        & hsv2rgb       \\
    palette         & sprintf       & stringcol.    & strlen        \\
    strstrt         & substr        & str[f|p]time  & system        \\
    timecol.        & tm\_hour      & tm\_mday      & tm\_min       \\
    tm\_mon         & tm\_sec       & tm\_wday      & tm\_yday      \\
    tm\_year        & time          & trim          & valid         \\
    value           & voxel         & word[s]       \\
\end{tabular}}


%%%%%%%%%%%%%%%%%%%%%%%%%%%%%%%%%%%%%%%%%%
\subsection*{Stylizing: lines \& text}


%%%%%%%%%%%%%%%%%%%%%%%%%%%%%%%%%%%%%%%%%%
\subsection*{Plot Types (\say{styles})}
\code{plot <data> using \dots with <style>}\\
\textit{Each PE in a }\textbf{plot}\textit{ command admits an optional override to the default }<style>\textit{, which then renders an eponymous plot type. The PE\textquotesingle s associated }\textbf{using}\textit{ must yield an acceptable number of data fields, as indicated next each }<style>\textit{ below.}

{\footnotesize
\begin{itemize}
    \item \textbf{arrows} \dots x, y, length, angle
    \item \textbf{boxerrorbars} \dots 
    \item \textbf{boxes} \dots x,y,[xwidth]
    \item \textbf{boxes} (3d) \dots x, y, z, [xwidth], [color]
    \item \textbf{boxplot} \dots x,y,[?,?]
    \item \textbf{boxxyerror} \dots x, y, [x$\delta$,y$\delta$]|[xmin,xmax,ymin,ymax]
    \item \textbf{candlesticks} \dots x, min, wsk\_min, wsk\_max, max
    \item \textbf{circles} \dots x, y, [rad,[arc\_beg, [arc\_end]]], [col]
    \item \textbf{ellipses} \dots x, y, [major, [minor, [angle]]]
    \item \textbf{dots} \dots [x], y, [z]
    \item \textbf{filledcurves} \dots x, y, yerror
    \item \textbf{financebars} \dots date, open, low, high, close
    \item \textbf{histeps} \dots [x], y, [z]
    \item \textbf{histogram} \dots y, [yerr|[ymin, ymax]] 
    \item \textbf{image} \dots bitmap-image 
    \item \textbf{impulses} \dots [x], y, [z] 
    \item \textbf{labels} \dots x, y, [z], string 
    \item \textbf{lines} \dots [x], y, [z]
    \item \textbf{linespoints} \dots [x], y, [z]
    \item \textbf{parallelaxes} \dots (\textit{one per axis}) 
    \item \textbf{polar}  \dots angle, radius 
    \item \textbf{points} \dots [x], y, [z]
    \item \textbf{polygons} \dots <\href{http://gnuplot.sourceforge.net/demo\_5.5/polygons.html}{polygon}>
    \item \textbf{spiderplot} \dots (\textit{one per axis}) 
    \item \textbf{{[f|fill|]}steps} \dots x, y
    \item \textbf{rgb[alpha|image]} \dots (see image)
    \item \textbf{vectors} \dots x, y, [z], x$\Delta$, y$\Delta$, [z$\Delta$] 
    \item \textbf{{[x|xy|y]}errorbars} \dots x, y, [x$\Delta$ | [xlow, xhigh]] 
    \item \textbf{{[x|xy|y]}errorlines} \dots x, y, [x$\Delta$ | [xlow, xhigh]] 
    \item \textbf{pm3d} \dots \textit{(see documentation)}
    \item \textbf{isosurface} \dots <voxel-grid-file>
    \item \textbf{zerrorfill} \dots x, y, z, [z$\Delta$ | [zlow, zhigh]] 
\end{itemize}}

\textit{$\exists$ variations on a theme, eg to create a \say{bee swarm} plot, use }\textbf{set jitter}\textit{, then }\textbf{with points}.\textit{ Similarly, for \say{fence plots} use the }\textbf{zerrorfill}\textit{ style. Some styles admin an additional qualifier, eg }\textbf{histogram }.\\


%%%%%%%%%%%%%%%%%%%%%%%%%%%%%%%%%%%%%%%%%%
\subsection*{Terminals (Output)}

\code{set term [terminal-name] [term-option]*}\\

\textit{Output can be rendered as code for external compilation (eg, tikz, svg, HTML canvas, etc); as image binaries (jpeg, gif, pdf, etc); within a specified terminal (qt, x11, etc); or directly to a supported printer (eg, epson, hp, etc). My preferred terminals are: cairolatex, canvas, \href{http://gnuplot.sourceforge.net/docs/tutorial.pdf}{epslatex}, gif, pdfcairo, pngcairo, pstricks, svg, tikz, wxt, x11. For example:}

\entry{35mm}{set output 'file.png'}{save to file}\\
\entry{45mm}{set term pngcairo size 400,600}{png}\\

\textit{See more examples \href{http://www.gnuplotting.org/output-terminals/}{here}. Each terminal has its own [unfortunately] distinct interface to control options such as the following (more common ones):}\\
%: font, scaling, transparency, background color, line width, paper size, etc. Most support the
{\footnotesize
\begin{tabular}{l l}
background <color>  & color | monochrome \\
{[no]}enhanced      &  font <fontname[,size]> \\
fontscale <scale>   &  {[no]}header <header> \\
{[input|standalone]}&  {[no]}inverted \\
linewidth <lw>      &  {[rounded|butt|square]} \\
title <title>       &  {[no]}transparent \\
size <XX>,<YY>      &  resolution <dpi> \\
\end{tabular}
}

%%%%%%%%%%%%%%%%%%%%%%%%%%%%%%%%%%%%%%%%%%
\subsection*{Command-line Shortcuts}

{\footnotesize
\begin{tabular}{c c | c c}
    \textasciicircum B & back char & \textasciicircum F & forwd char \\
    \textasciicircum A & begin line & \textasciicircum E & end line \\ 
    \textasciicircum H & del prev char & DEL & del curr char \\
    \textasciicircum D & del curr char & \textasciicircum K & del to EOL \\
    \textasciicircum L & redraw line & \textasciicircum U & delete line \\
    \textasciicircum W & del prev word & \textasciicircum V & inhibit \\
    TAB & file-complet\textquotesingle n & \textasciicircum P & back history \\
    \textasciicircum N & forward hist & \textasciicircum R & back-search \\
\end{tabular}}








%%%%%%%%%%%% BONEYARD %%%%%%%%%%%%%%

%%%%%%%%%%%%%%%%%%%%%%%%%%%%%%%%%%%%%%%%%%
%\subsection*{Fonts}
%%%%%%%%%%%%%%%%%%%%%%%%%%%%%%%%%%%%%%%%%%
%\subsection*{Strings}
%%%%%%%%%%%%%%%%%%%%%%%%%%%%%%%%%%%%%%%%%%
%\subsection*{Stylizing}
%%%%%%%%%%%%%%%%%%%%%%%%%%%%%%%%%%%%%%%%%%
%\subsection*{Data Input}
%%% Herefile
%\entry{35mm}{plot \textquotesingle -\textquotesingle \, <{}< EOD \dots EOD}{\say{here-document}}


