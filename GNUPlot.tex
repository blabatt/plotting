\section{GNUPlot}
\textit{General script pattern involves an arbitrary number of }\textbf{set}\textit{ commands for setup of axes, titles, fonts, and other styles; followed by one or more }\textbf{plot}\textit{ commands, each of which renders output to the default or designated \say{terminal}. See \href{http://gnuplot.info/demos/}{here} for numerous demos, or in <src>/demo directory.}


%%%%%%%%%%%%%%%%%%%%%%%%%%%%%%%%%%%%%%%%%%
\subsection*{Invocation}
\code{gnuplot [options]* [script-file]?}\\
\textit{GNUPlot is invoked in either batch mode (script given to STDIN); or in interactive mode, which exposes a prompt. Command line options include:}\
\begin{itemize}[label=-]
    \item V \quad \ul{version}
    \item h \quad \ul{help}
    \item p \quad \ul{persist} plot windows after close 
    \item d \quad \ul{default-settings}
    \item s \quad \ul{slow}: wait for font initialization
    \item e \quad execute given command
    \item c \quad call given following script
\end{itemize}
\textit{Default settings can be overridden at startup by initialization files in either .gnuplotrc (location determined at install) or \textasciitilde /.gnuplot. These files can invoke normal }\textbf{set}\textit{ commands as well as setting environment variables such as: } \texttt{GNUTERM}, \texttt{GDFONTPATH}, \texttt{GNUPLOT\_DEFAULT\_GDFONT}. \\


%%%%%%%%%%%%%%%%%%%%%%%%%%%%%%%%%%%%%%%%%%
\subsection*{Plotting} 
\code{[s]plot [<\ul{p}lot\_\ul{e}lement>]*}
%\code{[<data> using <map> with <style>]*}

\textit{PE has the following structure:}\\
\begin{itemize}[label=|]
\item axes
\item data
    \begin{itemize}
	\item bins
	\item csv
	\item index
	\item skip
	\item using
	    \begin{itemize}
		\item key
		\item {[x|x2|y|y2|cb]}ticlabels
	    \end{itemize}
    \end{itemize}
\item title
\item with
\end{itemize}


\textbf{replot} \textit{ and }\textbf{refresh}\textit{ repeat the last }\textbf{plot}\textit{ command. }\textbf{using}\textit{ phrase maps }<data>\textit{ to the abscissa \& ordinate (or angle and radius) as well as \say{plot style}-specific characteristics.}


\textit{Plot examples (multiple PEs, for clause, with clause, ranges, href{http://www.gnuplot.info/demo/param.html}{parametric}, various PE \& using arguments):}\\
\code{plot sin(x), cos(x)}\\
\code{plot for \dots}\\
\code{plot \textquotedbl file\textquotedbl with lines,}\\
\code{\phantom{xxx}\textquotedbl file2\textquotedbl with points}\\
\code{plot [-pi:pi*2] tan(x)}\\
\code{plot [t=1:10] [-pi:pi*2] tan(t)}\\
\code{plot \textquotedbl file\textquotedbl using (tan(\$2)):(\$3/\$4) smooth csplines \textbackslash}\\
\code{\phantom{xxx}axes x1y2 title \textquotedbl parametric\textquotedbl with lines}

\textit{Using examples:}\\

\entry{40mm}{\dots using 1:(\$2+\$3)}{derived fields}\\
\entry{40mm}{\dots using 1:5 \textquotesingle \%lf,\%lf,\%lf\textquotesingle}{comma separated}\\
\entry{40mm}{\dots using 1:(\$3>10 ? \$2 : 1/10)}{ternary filtered}\\
\entry{40mm}{\dots using 2:4:xtic(1):ytic(3)}{labels}\\



%%%%%%%%%%%%%%%%%%%%%%%%%%%%%%%%%%%%%%%%%%
\subsection*{Set Command}


%%%%%%%%%%%%%%%%%%%%%%%%%%%%%%%%%%%%%%%%%%
\subsection*{Script Syntax}

%%% Expressions
%%% Iterations
%%% Strings
%%% Command line substitutions

\entry{35mm}{\# \dots}{comment}



%%%%%%%%%%%%%%%%%%%%%%%%%%%%%%%%%%%%%%%%%%
\subsection*{Data Input}

%%% Herefile
\entry{35mm}{plot \textquotesingle -\textquotesingle \, <{}< EOD \dots EOD}{\say{here-document}}


%%%%%%%%%%%%%%%%%%%%%%%%%%%%%%%%%%%%%%%%%%
\subsection*{Built-in Functions}


%%%%%%%%%%%%%%%%%%%%%%%%%%%%%%%%%%%%%%%%%%
\subsection*{Stylizing: lines \& text}


%%%%%%%%%%%%%%%%%%%%%%%%%%%%%%%%%%%%%%%%%%
\subsection*{Plot Types}
\code{plot <data> using \dots with [style]}\\
\textit{Generally the \say{plot type} is indicated by setting a \say{style} per the above }\textbf{plot}\textit{ syntax. The following styles render eponymous plot types. In each case, }\textbf{using}\textit{ must yield an acceptable number of data fields, as indicated below.}

{\footnotesize
\begin{itemize}
    \item arrows \quad x, y, length, angle
    \item boxerrorbars \quad 
    \item boxes \quad x,y,[xwidth]
    \item boxes (3d) \quad x, y, z, [xwidth], [color]
    \item boxplot \quad x,y,[?,?]
    \item boxxyerror \quad x, y, [x$\delta$,y$\delta$]|[xmin,xmax,ymin,ymax]
    \item candlesticks \quad x, min, wsk\_min, wsk\_max, max
    \item circles \quad x, y, [rad,[arc\_beg, [arc\_end]]], [col]
    \item ellipses \quad x, y, [major, [minor, [angle]]]
    \item dots \quad [x], y, [z]
    \item filledcurves \quad x, y, yerror
    \item financebars \quad date, open, low, high, close
    \item histeps \quad [x], y, [z]
    \item histogram \quad y, [yerr|[ymin, ymax]] 
    \item image \quad bitmap-image 
    \item impulses \quad [x], y, [z] 
    \item labels \quad x, y, [z], string 
    \item lines \quad [x], y, [z]
    \item linespoints \quad [x], y, [z]
    \item parallelaxes \quad (\textit{one per axis}) 
    \item polar  \quad angle, radius 
    \item points \quad [x], y, [z]
    \item polygons \quad <\href{http://gnuplot.sourceforge.net/demo\_5.5/polygons.html}{polygon}>
    \item spiderplot \quad (\textit{one per axis}) 
    \item {[f|fill|]}steps \quad x, y
    \item rgb[alpha|image] \quad (see image)
    \item vectors \quad x, y, [z], x$\Delta$, y$\Delta$, [z$\Delta$] 
    \item {[x|xy|y]}errorbars \quad x, y, [x$\Delta$ | [xlow, xhigh]] 
    \item {[x|xy|y]}errorlines \quad x, y, [x$\Delta$ | [xlow, xhigh]] 
    \item pm3d \quad \textit{(see documentation)}
    \item isosurface \quad <voxel-grid-file>
    \item zerrorfill \quad x, y, z, [z$\Delta$ | [zlow, zhigh]] 
\end{itemize}}

\textit{There are a number of variations on a theme, eg to create a \say{bee swarm} plot, use }\textbf{with points}\textit{ and }\textbf{set jitter}.\textit{Similarly, for \say{fence plots} use the }\textbf{zerrorfill}\textit{ style. Some allow an additional qualifier, eg }\textbf{histogram [clustered|errorbars|rowstacked|columnstacked]}\textit{. In most cases, additional setup (as detailed above) is desired, eg:}\\
\code{set spiderplot; set paxis 1 range [0:100]}\\


%%%%%%%%%%%%%%%%%%%%%%%%%%%%%%%%%%%%%%%%%%
\subsection*{Terminals (Output)}

\code{set term [terminal-name] [term-option]*}\\

\textit{Output can be rendered as code for external compilation (eg, tikz, svg, HTML canvas, etc); as image binaries (jpeg, gif, pdf, etc); within a specified terminal (qt, x11, etc); or directly to a supported printer (eg, epson, hp, etc). My preferred terminals are: cairolatex, canvas, gif, pdfcairo, pngcairo, pstricks, svg, tikz, wxt, x11. For example:}

\entry{35mm}{set output 'file.png'}{save to file}\\
\entry{45mm}{set term pngcairo size 400,600}{png}\\

\textit{See more examples \href{http://www.gnuplotting.org/output-terminals/}{here}. Each terminal has its own [unfortunately] distinct interface to control options such as the following (more commonly implemented ones):}\\
%: font, scaling, transparency, background color, line width, paper size, etc. Most support the

\begin{itemize}
    \item background <color>
    \item color | monochrome
    \item {[no]}enhanced
    \item font <fontname[,size]>
    \item fontscale <scale>
    \item {[no]}header <header>
    \item {[input|standalone]}
    \item {[no]}inverted
    \item linewidth <lw>
    \item {[rounded|butt|square]}
    \item title <title>
    \item {[no]}transparent
    \item size <XX>,<YY>
    \item resolution <dpi>
    %\item charset [ascii|blocks|unicode]
\end{itemize}


\begin{comment}
\begin{tabular}{c c c c}
    aifm & aqua & be & caca \\
    cairolatex & canvas & cgm & context \\
    corel & debug & domterm & dumb \\
    dxf & dxy800a & eepic & emf \\
    emxvga & epscairo & epslatex & epson\_180dpi \\
    excl & fig & ggi & gif \\
    gpic & grass & hp2623a & hp2468 \\
    hp500c & hpgl & hpljii & hppj \\
    imagen & jpeg & kyp & latex \\
    linux & lua & mf & mif \\
    mp & pbm & pcl5 & pdf \\
    pdfcairo & pict2e & pm & png \\
    pngcairo & postscript & pslatex & pstricks \\
    qms & qt & regis & sixelgd \\
    svg & svga & tek40 & tek410x \\
    texdraw & tgif & tikz & tkcanvas \\
    tpic & vws & windows & wxt \\
    x11 & xlib
\end{tabular}
\end{comment}


%%%%%%%%%%%%%%%%%%%%%%%%%%%%%%%%%%%%%%%%%%
\subsection*{Command-line Shortcuts}
\begin{tabular}{c c | c c}
    \textasciicircum B & back char & \textasciicircum F & forwd char \\
    \textasciicircum A & begin line & \textasciicircum E & end line \\ 
    \textasciicircum H & del prev char & DEL & del curr char \\
    \textasciicircum D & del curr char & \textasciicircum K & del to EOL \\
    \textasciicircum L & redraw line & \textasciicircum U & delete line \\
    \textasciicircum W & del prev word & \textasciicircum V & inhibit \\
    TAB & file-complet\textquotesingle n & \textasciicircum P & back history \\
    \textasciicircum N & forward hist & \textasciicircum R & back-search \\
\end{tabular}








%%%%%%%%%%%% BONEYARD %%%%%%%%%%%%%%

%%%%%%%%%%%%%%%%%%%%%%%%%%%%%%%%%%%%%%%%%%
%\subsection*{Fonts}
%%%%%%%%%%%%%%%%%%%%%%%%%%%%%%%%%%%%%%%%%%
%\subsection*{Strings}
%%%%%%%%%%%%%%%%%%%%%%%%%%%%%%%%%%%%%%%%%%
%\subsection*{Stylizing}


